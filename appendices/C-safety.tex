% Appendix AC: Safety Guidelines

\section{Safety Guidelines}

These guidelines will be enforced on all applicable hardware; it is the responsibility of the demonstrating team to be informed of them and of their system adhering to them reliably and with a reasonable tolerance.

\subsection{Electrical Safety}

\subsubsection{Voltage Classifications}
All voltages in nominal RMS.
\begin{enumerate}[noitemsep]
    \item \textbf{Low Voltage A (LVA)}: <50V AC, <120V DC
    \item \textbf{Low Voltage B (LVB)}: 50-1000V AC, 120-1000V DC
    \item \textbf{High Voltage A (HVA)}: 1kV-50kV
    \item \textbf{High Voltage B (HVB)}: >50kV
\end{enumerate}

\subsubsection{Required Safety Measures}
\begin{enumerate}[noitemsep]
    \item All circuits must have overcurrent protection
    \item Emergency stop buttons accessible within 1 meter
    \item Lockout/tagout procedures for maintenance  
    \item Ground fault circuit interrupters for portable equipment
    \item Frequency-dependent grounding (see Section~\ref{sec:grounding})
    \item Insulated tools for work above 50V
    \item Clear labeling of voltage levels
    \item Arc flash analysis required for systems >240V
\end{enumerate}

\subsubsection{Grounding Requirements}
\label{sec:grounding}
\begin{enumerate}[noitemsep]
    \item \textbf{DC Systems}: <0.1$\Omega$ to earth ground
    \item \textbf{<1 MHz}: Single-point grounding
    \item \textbf{1-10 MHz}: Transition zone, paths <$\lambda$/20
    \item \textbf{>10 MHz}: Multi-point grounding
    \item \textbf{RF Systems}: Wide copper straps, not round wires
    \item \textbf{Klystrons}: Separate RF and DC grounds, connected at one point
\end{enumerate}

\subsubsection{Personal Protective Equipment}
\begin{enumerate}[noitemsep]
    \item Medium Voltage: Class 0 gloves + 8 cal/cm$^2$ arc suit (NFPA 70E)
    \item High Voltage: Class 2 gloves + calculated arc rating (IEEE 1584)
    \item Safety glasses or face shield (ANSI Z87.1)
    \item Insulated footwear for HV work
    \item Remove conductive jewelry
\end{enumerate}

\subsection{RF Safety}

\subsubsection{Exposure Limits}
IEEE C95.1-2019 standard applies:
\begin{enumerate}[noitemsep]
    \item \textbf{30-100 MHz}: 2 W/m$^2$
    \item \textbf{100-400 MHz}: f/50 W/m$^2$ (where f is in MHz)
    \item \textbf{400-2000 MHz}: 10 W/m$^2$
    \item \textbf{2-300 GHz}: 10 W/m$^2$ (reduced from 20 W/m$^2$)
    \item Time-averaged over 6 minutes for pulsed operation
\end{enumerate}

\subsubsection{RF Safety Measures}
\begin{enumerate}[noitemsep]
    \item RF leakage measurements required before operation
    \item Shielded enclosures for high-power systems
    \item Interlocks on access doors
    \item RF warning signs and lights
    \item Personal RF monitors for operators
    \item Minimum 2-meter public exclusion zone
\end{enumerate}

\subsection{Radiation Safety}

\subsubsection{Dose Limits}
Following ICRP 103 recommendations:
\begin{enumerate}[noitemsep]
    \item \textbf{Radiation Workers}: 20 mSv/year
    \item \textbf{General Public}: 1 mSv/year
    \item \textbf{Controlled Areas}: >0.1 mSv/week (NCRP 144)
    \item \textbf{Design Goal}: <0.5 mSv/year in controlled areas (NCRP 147)
\end{enumerate}

\subsubsection{Radiation Monitoring for Demonstrations}
\begin{enumerate}
    \item Pre-operation radiation survey required with documented baseline
    \item Continuous monitoring during operation with a visible display
    \item Maximum dose rates:
    \begin{enumerate}
        \item Public areas: 2.5 $\mu$Sv/h
        \item Operator position: 10 $\mu$Sv/h
        \item Boundary: 0.5 $\mu$Sv/h
    \end{enumerate}
    \item Automatic shutdown if any limit is exceeded
    \item Post-operation survey to verify no activation
    \item All readings documented and retained for 5 years
\end{enumerate}

\subsubsection{Activation Products}
\begin{enumerate}[noitemsep]
    \item \textbf{Be-7}: 1 year cooling period required
    \item \textbf{Na-22}: 10 years cooling period required
    \item Storage and handling per IAEA SSG-59 guidelines
\end{enumerate}

\subsubsection{Monitoring Requirements}
\begin{enumerate}[noitemsep]
    \item Personal dosimetry: Monthly badge exchange (10 CFR 20)
    \item Area monitors in all controlled areas
    \item Extremity badges for hands-on work
    \item Survey meters available during operations
\end{enumerate}

\subsection{Vacuum Safety}

\subsubsection{Implosion Hazards}
\begin{enumerate}[noitemsep]
    \item Glass viewports must be protected with mesh or shields
    \item Vacuum chambers must have pressure relief devices
    \item No modifications to certified vacuum vessels
    \item Regular inspection for fatigue or damage
\end{enumerate}

\subsubsection{Pump Safety}
\begin{enumerate}[noitemsep]
    \item Mechanical pumps: exhaust vented properly
    \item Turbomolecular pumps: secured against rotor failure
    \item Oil-free pumps preferred for public demonstrations
    \item Pressure gauges must be operational
    \item Oxygen monitoring for enclosed spaces
\end{enumerate}

\subsection{Cryogenic Safety}

\subsubsection{Liquid Nitrogen Handling}
\begin{enumerate}[noitemsep]
    \item Maximum 5 liters for demonstrations
    \item Appropriate Dewar vessels only
    \item -196°C rated gloves + face shield (ANSI Z87.1)
    \item Adequate ventilation to prevent oxygen displacement
    \item Oxygen monitors required in confined spaces
    \item No sealed containers
\end{enumerate}

\subsection{Laser Safety}

\subsubsection{Classifications and Requirements}
\begin{enumerate}[noitemsep]
    \item Class 1: Safe under all conditions
    \item Class 2: Safe for momentary viewing (<0.25s)
    \item Class 3R: Restricted, requires controls
    \item Class 3B and 4: Not permitted in demonstrations
\end{enumerate}

\subsubsection{Control Measures}
\begin{enumerate}[noitemsep]
    \item Beam stops required
    \item No specular reflective surfaces in beam path
    \item Warning signs and indicator lights
    \item Protective eyewear if Class 3R used
\end{enumerate}

\subsection{Emergency Procedures}

\subsubsection{Incident Command System}
All emergencies managed through defined roles:
\begin{enumerate}[noitemsep]
    \item \textbf{Safety Officer}: Scene commander
    \item \textbf{Technical Lead}: Equipment shutdown
    \item \textbf{Medical Officer}: Casualty care
    \item \textbf{Communications}: External liaison
\end{enumerate}

\subsubsection{Electrical Shock}
\begin{enumerate}[noitemsep]
    \item Do not touch victim if still in contact with source
    \item De-energize circuit at main disconnect
    \item Call emergency services (112 in EU)
    \item Begin CPR if trained and victim unresponsive
    \item Use AED if available
    \item Document incident within 48 hours
\end{enumerate}

\subsubsection{RF Exposure}
\begin{enumerate}[noitemsep]
    \item Immediately shut down RF source
    \item Remove person from exposure area
    \item Seek medical evaluation even if asymptomatic
    \item Document exposure level and duration
    \item Report per incident investigation procedures
\end{enumerate}

\subsubsection{Vacuum Failure}
\begin{enumerate}[noitemsep]
    \item Move away from vacuum vessel
    \item Shut down pumps if safe to do so
    \item Ventilate area if oil mist present
    \item Check for injuries from flying debris
    \item Initiate incident investigation
\end{enumerate}

\subsection{Training Requirements}

\subsubsection{Minimum Training for Operations}
At least one of the operators must comply with one of the trainings, except for the general electrical safety, which all operators must be trained on.
\begin{enumerate}[noitemsep]
    \item General electrical safety: 4 hours + competency test
    \item System-specific training: 8 hours + hands-on evaluation
    \item Radiation safety: 16 hours + written and practical exam
    \item Emergency response procedures: 4 hours + drill participation
    \item Annual refresher training required
    \item \textbf{Total minimum}: 40 hours with assessment
\end{enumerate}

\subsubsection{Documentation}
\begin{enumerate}[noitemsep]
    \item Training records with test scores maintained permanently
    \item Risk assessment per ISO 31000 before operations
    \item Arc flash analysis (updated every 5 years)
    \item Shielding verification (annual)
    \item Operating procedures posted at demonstration
    \item Emergency contact numbers displayed
    \item Incident investigation reports within 48 hours
\end{enumerate}

\subsection{Safety Management System}

\subsubsection{Required Elements}
\begin{enumerate}[noitemsep]
    \item Written safety program document
    \item Hazard identification and risk assessment
    \item Safety training program with competency verification
    \item Incident investigation procedures with root cause analysis
    \item Corrective action tracking system
    \item Regular safety audits and inspections
    \item Management review and continuous improvement
\end{enumerate}