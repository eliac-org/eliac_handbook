% Appendix AB: Technical Glossary

\section{Technical Glossary}

\subsection{Accelerator Components}

\begin{description}
\item[Accelerating Cavity] Resonant metallic structure where electromagnetic fields accelerate particles. Can be normal-conducting or superconducting.

\item[Beam Position Monitor (BPM)] Diagnostic device that measures the transverse position of the particle beam within the beampipe.

\item[Buncher] RF cavity operating at low energy to compress particle bunches longitudinally before main acceleration.

\item[Cathode] Electron-emitting surface in an electron gun. Can be thermionic (heated), photocathode (laser-driven), or field emission.

\item[Coupler] Device for transferring RF power from transmission line into accelerating cavity. Can be capacitive, inductive, or waveguide-based.

\item[Electron Gun] Source of electrons for acceleration. Includes cathode, extraction electrode, and focusing elements.

\item[Faraday Cup] Simple beam diagnostic that collects all beam current for measurement.

\item[Klystron] High-power RF amplifier tube commonly used to drive accelerating structures.

\item[Quadrupole] Magnet with four poles used for transverse beam focusing. Focuses in one plane while defocusing in the other.

\item[RF Window] Vacuum barrier transparent to RF power, typically made of ceramic or sapphire.

\item[Solenoid] Cylindrical coil producing longitudinal magnetic field for beam focusing, especially at low energy.

\item[Waveguide] Hollow metallic tube for RF power transmission with low loss.
\end{description}

\subsection{Performance Metrics}

\begin{description}
\item[Beam Loading] Reduction in cavity voltage due to energy extraction by the beam.

\item[Duty Cycle] Ratio of pulse duration to repetition period for pulsed operation.

\item[Emittance] Measure of beam quality in phase space. Product of beam size and angular divergence. Units: mm-mrad.

\item[Energy Spread] RMS or FWHM variation in particle energies within the beam. Expressed as percentage or absolute value.

\item[Gradient] Accelerating electric field strength, typically expressed in MV/m. Key figure of merit for accelerating structures.

\item[Quality Factor (Q)] Ratio of stored energy to energy loss per RF cycle. Indicates cavity efficiency.

\item[Repetition Rate] Frequency of beam pulses, typically in Hz.

\item[Shunt Impedance] Measure of cavity efficiency relating voltage gain to power consumption. Units: M$\Omega$/m.

\item[Transit Time Factor] Efficiency factor accounting for particle velocity change during acceleration.

\item[Wake Field] Electromagnetic field left behind by beam that can affect following particles.
\end{description}

\subsection{Simulation Terms}

\begin{description}
\item[Convergence Study] Systematic refinement of mesh or parameters to ensure simulation accuracy.

\item[Eigenmode] Natural resonant mode of electromagnetic cavity.

\item[Mesh] Discretization of geometry for numerical simulation.

\item[Particle-In-Cell (PIC)] Simulation method tracking individual particles in self-consistent fields.

\item[S-Parameters] Scattering parameters describing RF network behavior.

\item[Space Charge] Collective electromagnetic field of beam affecting particle dynamics.
\end{description}

\subsection{Operational Terms}

\begin{description}
\item[Conditioning] Process of gradually increasing cavity field to eliminate field emission sites.

\item[Interlock] Safety system preventing operation under unsafe conditions.

\item[Multipacting] Resonant secondary electron emission limiting achievable field.

\item[Quench] Loss of superconductivity due to heating or excessive field.

\item[Vacuum Breakdown] Electrical discharge in vacuum at high field strength.
\end{description}