% Section E: Evaluation

\renewcommand{\thesection}{E}
\section{Evaluation}

\subsection{Scoring Framework}

\subsubsection{Component Scoring Distribution}
\begin{tabular}{lc}
\toprule
Component & Weight \\
\midrule
Electron Source & 200 \\
Acceleration Section & 300 \\
RF System & 200 \\
Vacuum \& Diagnostics & 100 \\
Integration \& Controls & 100 \\
Innovation Bonus & 100 \\
\midrule
\textbf{Total} & \textbf{1000} \\
\bottomrule
\end{tabular}

\subsubsection{Evaluation Criteria per Component}
Each component evaluated on five dimensions:
\begin{enumerate}[noitemsep]
    \item \textbf{Physics Accuracy} (30\%): Correct application of principles, valid assumptions
    \item \textbf{Design Completeness} (25\%): All necessary elements addressed
    \item \textbf{Simulation Quality} (20\%): Convergence, validation, parameter studies
    \item \textbf{Innovation} (15\%): Creative solutions and novel approaches
    \item \textbf{Documentation} (10\%): Clarity, completeness, reproducibility
\end{enumerate}

\subsection{Innovation Weighting}

\subsubsection{}
Innovation scoring rewards novel approaches over standard solutions:

\begin{enumerate}[noitemsep]
    \item \textbf{90-100 points}: Completely novel approach, potentially publishable
    \item \textbf{70-80 points}: Significant modification of existing concepts
    \item \textbf{50-60 points}: Creative combination of known techniques
    \item \textbf{30-40 points}: Minor variations on standard designs
    \item \textbf{0-20 points}: Direct copy of existing accelerator
\end{enumerate}

\subsubsection{}
Teams copying existing designs without innovation will lose against lesser-performing novel ideas. Innovation must be justified with clear advantages or unique applications.

\subsection{Jury Composition}

\subsubsection{}
Jury members must declare conflicts of interest. Members with direct connections to participating teams recuse from evaluating those teams.

\subsection{Evaluation Process}

\subsubsection{Document Review}
Technical papers evaluated before event using standardized rubrics. Each paper reviewed by minimum three jury members independently.

\subsubsection{Presentation Assessment}
Live presentations evaluated for:
\begin{enumerate}[noitemsep]
    \item Technical content clarity (40\%)
    \item Response to questions (30\%)
    \item Visual aids quality (15\%)
    \item Time management (15\%)
\end{enumerate}

\subsubsection{Final Scoring}
\begin{enumerate}[noitemsep]
    \item Technical paper: 1000 of total score
    \item Oral presentation: 300 of total score
    \item Poster quality: 100 of total score
\end{enumerate}

\subsection{Awards and Recognition}

\subsubsection{Competition Awards}
\begin{enumerate}[noitemsep]
    \item \textbf{ELIAC Champion}: Highest overall score
    \item \textbf{Innovation Award}: Most novel technical approach
    \item \textbf{Best Newcomer}: Top score among first-time institutions
    \item \textbf{Component Excellence}: Best design in specific subsystem
    \item \textbf{People's Choice}: Voted by event participants
\end{enumerate}

\subsubsection{Special Mentions}
\begin{enumerate}[noitemsep]
    \item Best simulation methodology
    \item Most thorough validation
    \item Best safety analysis
    \item Outstanding presentation
    \item Best international collaboration
\end{enumerate}

\subsection{Peer Review Process}

\subsubsection{}
Post-event peer review ensures fairness and accuracy:

\begin{tabular}{ll}
\toprule
Timeline & Activity \\
\midrule
Days 1-7 & Papers available to volunteer reviewers \\
Days 8-9 & Review submissions compiled \\
Days 10-11 & Committee evaluates flagged issues \\
Day 12 & Final results published \\
Day 30 & Winners' papers submitted for magazine \\
Day 60 & ELIAC Magazine published \\
\bottomrule
\end{tabular}

\subsubsection{}
Reviewers may flag calculation errors, uncited prior art, safety concerns, or academic integrity issues. Committee investigates all flags before finalizing results.