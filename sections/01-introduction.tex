% Section I: Introduction

\renewcommand{\thesection}{I}
\section{Introduction}

\subsection{Purpose}

\subsubsection{}
The European Linear Accelerator Challenge (ELIAC) is the first international academic challenge focused on particle accelerators, specifically on the design and development of an electron Linear Accelerator (LINAC). This initiative aims to foster collaboration between teams from different countries and specialties, while facing the challenge of developing a particle acceleration system with potential applications in different industry and research fields.

\subsubsection{}
ELIAC is not limited to the development of a specific linear accelerator model, but also aims to strengthen international synergies in the academic and research fields, promoting teamwork, innovation, and knowledge exchange among students and professionals from different disciplines.

\subsection{Technical Approach}

\subsubsection{}
The electron linear accelerator (LINAC) was selected as the foundational model due to its relative structural simplicity and modular nature, characteristics that make it an ideal starting point for an academic competition of this kind.

\subsubsection{}
Unlike circular accelerators such as cyclotrons or synchrotrons, a LINAC is based on a linear design that facilitates scalability and allows for more precise control of the energy delivered to the particles.

\subsection{Educational Philosophy}

\subsubsection{}
The approach emphasizes a bottom-up development process, where degree and master's students take an active role in every stage: from electromagnetic design and particle dynamics simulation to RF component selection, vacuum system integration, and beam diagnostics.

\subsubsection{}
The educational objectives pursued are:
\begin{enumerate}[noitemsep]
    \item Development of technical skills in advanced simulation tools
    \item Hands-on learning through practical application
    \item Interdisciplinary collaboration experience
    \item Professional development in a research environment
\end{enumerate}

\subsection{Competition Scope}

\subsubsection{}
The competition focuses on the design and simulation of electron LINAC prototypes. Physical construction is encouraged for demonstration purposes but is not scored. Teams design complete systems while specializing in areas of strength.

\subsubsection{}
There are no constraints on beam energy or accelerator length, provided designs are safe and serve a stated objective. Teams must define a specific application context for their design.