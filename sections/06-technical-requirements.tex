% Section TR: Technical Requirements

\renewcommand{\thesection}{TR}
\section{Technical Requirements}

\subsection{Minimum Viable LINAC (MVL)}

\subsubsection{}
All teams must design a complete electron LINAC system. While complexity may vary, every submission must include all fundamental subsystems:

\begin{itemize}[noitemsep]
    \item \textbf{Electron Source} (20\% of score): Minimum DC thermionic gun design, advanced options include RF photocathode or field emission
    \item \textbf{Acceleration Section} (30\% of score): Minimum single cavity or drift tube, advanced multi-cell structures encouraged
    \item \textbf{RF System} (20\% of score): Minimum conceptual power coupling, complete LLRF and feedback systems preferred
    \item \textbf{Vacuum \& Diagnostics} (10\% of score): Basic pressure calculation and one diagnostic minimum
    \item \textbf{Integration \& Controls} (10\% of score): Block diagram minimum, full control system design encouraged
    \item \textbf{Innovation Bonus} (10\% of score): Novel approaches in any subsystem
\end{itemize}

\subsubsection{}
Teams weak in one area may compensate with excellence in others, but all subsystems must be addressed to qualify.

\subsection{Software Requirements}

\subsubsection{Recommended Open Source Tools}
\begin{itemize}[noitemsep]
    \item \textbf{EM Design}: openEMS, MEEP, or equivalent
    \item \textbf{Beam Dynamics}: ASTRA, elegant, OPAL
    \item \textbf{Particle Transport}: GEANT4 (mandatory for radiation studies)
    \item \textbf{Electronics}: KiCad (mandatory for PCB design)
    \item \textbf{Thermal/Structural}: CalculiX, Code\_Aster
    \item \textbf{Vacuum}: MOLFLOW+ for pressure distribution
\end{itemize}

\subsubsection{Accepted Commercial Tools}
\begin{itemize}[noitemsep]
    \item \textbf{EM Design}: CST Studio Suite, ANSYS HFSS, COMSOL
    \item \textbf{Beam Dynamics}: GPT (General Particle Tracer), TRACK
    \item \textbf{Electronics}: Altium
    \item \textbf{Thermal/Structural}: ANSYS, COMSOL
\end{itemize}

Teams using commercial software must provide version numbers and verify license validity.

\subsection{Validation Requirements}

\subsubsection{}
All teams must demonstrate software competence and simulation accuracy through standardized benchmarks:

\begin{enumerate}[noitemsep]
    \item \textbf{Standard Cavity Test}: Simulate a published reference cavity (e.g., 1.3 GHz TESLA cell) and show agreement within 5\% of published parameters
    \item \textbf{Beam Transport Validation}: Model simple drift space focusing and compare with analytical solutions
    \item \textbf{Convergence Studies}: Demonstrate mesh independence for all electromagnetic simulations
    \item \textbf{Parameter Sensitivity}: Show design robustness through parameter sweeps
\end{enumerate}

\subsubsection{}
Validation results must be included as appendix in technical paper. Inadequate validation subject to point deductions.

\subsection{Documentation Standards}

\subsubsection{Technical Paper Requirements}
\begin{enumerate}[noitemsep]
    \item \textbf{Abstract} (200-300 words): Clear statement of objectives and achievements
    \item \textbf{Introduction \& Objectives} (1-2 pages): Context and specific goals
    \item \textbf{Theoretical Background} (2-3 pages): Relevant physics and engineering principles
    \item \textbf{System Design} (6-10 pages): Detailed description of all subsystems
    \item \textbf{Simulation Methodology} (2-3 pages): Software tools, settings, validation
    \item \textbf{Results \& Analysis} (3-5 pages): Performance metrics and optimization
    \item \textbf{Safety \& Fabrication} (1-2 pages): Practical considerations
    \item \textbf{Conclusions}: Summary and future work
    \item \textbf{References}: IEEE format required
\end{enumerate}

\subsubsection{Format Specifications}
\begin{itemize}[noitemsep]
    \item \textbf{Length}: 15-25 pages excluding references and appendices
    \item \textbf{Format}: A4 paper, 11pt font, 1.5 line spacing
    \item \textbf{Margins}: 25mm all sides
    \item \textbf{File}: PDF format, maximum 25MB
    \item \textbf{Language}: English (UK or US spelling acceptable)
    \item \textbf{Figures}: High resolution, properly captioned and referenced
\end{itemize}

\subsection{Design Constraints}

\subsubsection{}
While innovation is encouraged, designs must respect physical laws and engineering feasibility:
\begin{itemize}[noitemsep]
    \item Maximum surface electric field justified by breakdown studies
    \item Thermal loads addressed with cooling solutions
    \item Vacuum requirements achievable with specified pumping
    \item Power consumption within reasonable limits
    \item Materials commercially available
\end{itemize}

\subsubsection{}
Exotic acceleration schemes (plasma, dielectric, etc.) permitted but require thorough justification and complete system design including drivers.